% Options for packages loaded elsewhere
\PassOptionsToPackage{unicode}{hyperref}
\PassOptionsToPackage{hyphens}{url}
\PassOptionsToPackage{dvipsnames,svgnames,x11names}{xcolor}
%
\documentclass[
  12pt]{article}

\usepackage{amsmath,amssymb}
\usepackage{iftex}
\ifPDFTeX
  \usepackage[T1]{fontenc}
  \usepackage[utf8]{inputenc}
  \usepackage{textcomp} % provide euro and other symbols
\else % if luatex or xetex
  \usepackage{unicode-math}
  \defaultfontfeatures{Scale=MatchLowercase}
  \defaultfontfeatures[\rmfamily]{Ligatures=TeX,Scale=1}
\fi
\usepackage{lmodern}
\ifPDFTeX\else  
    % xetex/luatex font selection
\fi
% Use upquote if available, for straight quotes in verbatim environments
\IfFileExists{upquote.sty}{\usepackage{upquote}}{}
\IfFileExists{microtype.sty}{% use microtype if available
  \usepackage[]{microtype}
  \UseMicrotypeSet[protrusion]{basicmath} % disable protrusion for tt fonts
}{}
\makeatletter
\@ifundefined{KOMAClassName}{% if non-KOMA class
  \IfFileExists{parskip.sty}{%
    \usepackage{parskip}
  }{% else
    \setlength{\parindent}{0pt}
    \setlength{\parskip}{6pt plus 2pt minus 1pt}}
}{% if KOMA class
  \KOMAoptions{parskip=half}}
\makeatother
\usepackage{xcolor}
\setlength{\emergencystretch}{3em} % prevent overfull lines
\setcounter{secnumdepth}{5}
% Make \paragraph and \subparagraph free-standing
\makeatletter
\ifx\paragraph\undefined\else
  \let\oldparagraph\paragraph
  \renewcommand{\paragraph}{
    \@ifstar
      \xxxParagraphStar
      \xxxParagraphNoStar
  }
  \newcommand{\xxxParagraphStar}[1]{\oldparagraph*{#1}\mbox{}}
  \newcommand{\xxxParagraphNoStar}[1]{\oldparagraph{#1}\mbox{}}
\fi
\ifx\subparagraph\undefined\else
  \let\oldsubparagraph\subparagraph
  \renewcommand{\subparagraph}{
    \@ifstar
      \xxxSubParagraphStar
      \xxxSubParagraphNoStar
  }
  \newcommand{\xxxSubParagraphStar}[1]{\oldsubparagraph*{#1}\mbox{}}
  \newcommand{\xxxSubParagraphNoStar}[1]{\oldsubparagraph{#1}\mbox{}}
\fi
\makeatother


\providecommand{\tightlist}{%
  \setlength{\itemsep}{0pt}\setlength{\parskip}{0pt}}\usepackage{longtable,booktabs,array}
\usepackage{calc} % for calculating minipage widths
% Correct order of tables after \paragraph or \subparagraph
\usepackage{etoolbox}
\makeatletter
\patchcmd\longtable{\par}{\if@noskipsec\mbox{}\fi\par}{}{}
\makeatother
% Allow footnotes in longtable head/foot
\IfFileExists{footnotehyper.sty}{\usepackage{footnotehyper}}{\usepackage{footnote}}
\makesavenoteenv{longtable}
\usepackage{graphicx}
\makeatletter
\newsavebox\pandoc@box
\newcommand*\pandocbounded[1]{% scales image to fit in text height/width
  \sbox\pandoc@box{#1}%
  \Gscale@div\@tempa{\textheight}{\dimexpr\ht\pandoc@box+\dp\pandoc@box\relax}%
  \Gscale@div\@tempb{\linewidth}{\wd\pandoc@box}%
  \ifdim\@tempb\p@<\@tempa\p@\let\@tempa\@tempb\fi% select the smaller of both
  \ifdim\@tempa\p@<\p@\scalebox{\@tempa}{\usebox\pandoc@box}%
  \else\usebox{\pandoc@box}%
  \fi%
}
% Set default figure placement to htbp
\def\fps@figure{htbp}
\makeatother

\addtolength{\oddsidemargin}{-.5in}%
\addtolength{\evensidemargin}{-1in}%
\addtolength{\textwidth}{1in}%
\addtolength{\textheight}{1.7in}%
\addtolength{\topmargin}{-1in}%
\makeatletter
\@ifpackageloaded{caption}{}{\usepackage{caption}}
\AtBeginDocument{%
\ifdefined\contentsname
  \renewcommand*\contentsname{Table of contents}
\else
  \newcommand\contentsname{Table of contents}
\fi
\ifdefined\listfigurename
  \renewcommand*\listfigurename{List of Figures}
\else
  \newcommand\listfigurename{List of Figures}
\fi
\ifdefined\listtablename
  \renewcommand*\listtablename{List of Tables}
\else
  \newcommand\listtablename{List of Tables}
\fi
\ifdefined\figurename
  \renewcommand*\figurename{Figure}
\else
  \newcommand\figurename{Figure}
\fi
\ifdefined\tablename
  \renewcommand*\tablename{Table}
\else
  \newcommand\tablename{Table}
\fi
}
\@ifpackageloaded{float}{}{\usepackage{float}}
\floatstyle{ruled}
\@ifundefined{c@chapter}{\newfloat{codelisting}{h}{lop}}{\newfloat{codelisting}{h}{lop}[chapter]}
\floatname{codelisting}{Listing}
\newcommand*\listoflistings{\listof{codelisting}{List of Listings}}
\makeatother
\makeatletter
\makeatother
\makeatletter
\@ifpackageloaded{caption}{}{\usepackage{caption}}
\@ifpackageloaded{subcaption}{}{\usepackage{subcaption}}
\makeatother

\usepackage[]{natbib}
\bibliographystyle{agsm}
\usepackage{bookmark}

\IfFileExists{xurl.sty}{\usepackage{xurl}}{} % add URL line breaks if available
\urlstyle{same} % disable monospaced font for URLs
\hypersetup{
  pdftitle={Research on Research on Research: Analyzing historical trends in statistical and computational research from the 1990s to modern day},
  pdfauthor={Naren Prakash},
  pdfkeywords={retrospective analysis, arXiv, publication analysis,
forecasting, time series},
  colorlinks=true,
  linkcolor={blue},
  filecolor={Maroon},
  citecolor={Blue},
  urlcolor={Blue},
  pdfcreator={LaTeX via pandoc}}



\begin{document}


\def\spacingset#1{\renewcommand{\baselinestretch}%
{#1}\small\normalsize} \spacingset{1}


%%%%%%%%%%%%%%%%%%%%%%%%%%%%%%%%%%%%%%%%%%%%%%%%%%%%%%%%%%%%%%%%%%%%%%%%%%%%%%

\date{March 19, 2025}
\title{\bf Research on Research on Research: Analyzing historical trends
in statistical and computational research from the 1990s to modern day}
\author{
Naren Prakash\thanks{The author gratefully acknowledges Professor Lew
for their encouragement and advice regarding the production of this
paper.}\\
Department of Statistics and Data Science, University of California, Los
Angeles\\
}
\maketitle

\bigskip
\bigskip
\begin{abstract}
This paper aims to analyze the changes in research paper output for
different statistical and computational fields over the time period from
the 1990s to modern day (2025). The paper also projects short term
growth for recently emerging fields in an effort to predict the fields
that will receive further resource funding and attention in the near
future. The research papers used for this analysis are sourced from a
dataset of papers from the pre-print journal arXiv.
\end{abstract}

\noindent%
{\it Keywords:} retrospective analysis, arXiv, publication analysis,
forecasting, time series
\vfill

\newpage
\spacingset{1.9} % DON'T change the spacing!


\section{Introduction}\label{sec-intro}

In recent years, with the fields of artifical intelligence and machine
learning becoming important parts of the public lexicon and increasingly
becoming involved in our day to day lives, we've seen firsthand large
changes in statistical and computational research. With statistical
methods increasingly becoming intertwined with computational principles,
such as its integration with aspects of computer science, the future of
statistics and computation appear to be one and the same. How does this
current research landscape compare with that of the landscape a mere 30
years ago? This paper aims to analyze historical trends in statistical
and computational research, as tracked by papers submitted to the online
pre-print journal arXiv, in order to visualize the dramatic changes
we've seen over the years and find any subfields growing in the present
that could yet transform the landscape of the future. This analysis of
historical trends will be conducted using a specific dataset available
on Kaggle \citet{arXiv:kaggle:data}.

\section{Literature Review}\label{lit-review}

Looking at themes in statistical and computational research is nothing
new. For instance, \citet{gelm:veht:2021} analyzed the dominant
statistical ideas of the past 50 years, suggesting inferential methods,
computational algorithms, and data nalysis have been the most impactful
in the shifting of the research landscape. Smaller subsets of time have
also been analyzed, with \citet{jun:yoo:choi:2018} using Google Trends
to track the growth of different subfields of research (with an emphasis
on big data and application). This paper aims to look at a similar
problem with a different lens, using publication outputs themselves as a
way of analyzing changes in research focus and interest. In doing this,
the paper aims to also obtain an indication of the subfields with
increasing research interest in the short term that may lend itself to
future publications. Evaluating the research trends of the future has
often involved modeling itself, such as the hype cycle model
\citet{dedehayir:2016}. This model aims to track the life cycle of
technological innovations. In a similar vein, this paper aims to use
current and recent paper production output to indicate trends of the
near future. As the methodology involves using the research paper
pre-prints themselves, it may provide a clearer picture of specific
publication interests and trends rather than topics and concepts in
general.

\section{Research Questions}\label{sec-questions}

\begin{itemize}
\tightlist
\item
  What statistical and computational fields have seen the largest
  increase in publications?
\item
  How have the most published statistical and computational fields
  changed over time?
\item
  What statistical and computational fields are projected to grow the
  most in the coming years?
\end{itemize}

\section{Data}\label{sec-data}

\subsection{Data Description}\label{data-description}

The arXiv paper dataset consists of 136,238 observations and 10 columns.
The 10 columns present in the data are: id, title, category, category
code, published date, updated date, authors, first author, summary, and
summary word count. Only the summary word count is a numeric variable.
This data is scraped directly from arXiv, aiming to provide a
representative sample of research published on the platform.

\subsection{Data Pre-Processing}\label{data-pre-processing}

The original dataset will have its variables converted to categorical
variables for grouping and analytical purposes, with the exception of
the summary word count due to its numeric nature. Following this, two
selected lists of subtopics will be created for the purpose of data
partitioning and separate analysis.

\subsection{Data Partitioning}\label{data-partitioning}

While acknowledging the connected nature of statistical and
computational research in the present and future, this paper will
partition the data into two halves. One half will be comprised of
research deemed statistical in nature, and the other half will be
comprised of research deemed as computational. This split in the data is
done to narrow down the problem and allow for ease of analysis and
interpretation of the results. Along with this, the data will also be
subset in terms of time. In order to keep each yearly subset of papers
as a representative sample of all research output for that year, the
time range will be limited to exclude publication in 2025. This is done
so that the resulting analysis will focus on comparison with full yearly
samples of research data, rather than extrapolating from the research
output in the year 2025 as of now.

\section{Methods}\label{sec-meth}

\subsection{Quantifying growth}\label{quantifying-growth}

For the purposes of this paper, growth will be represented by two
metrics. Firstly, the simple percentage change from year to year for
each subfield will be considered. In addition to this, the proportion of
overall research represented by each subfield over time will also be
used to evaluate growth in research interest and output.

\subsection{Trend analysis and short-term
prediction}\label{trend-analysis-and-short-term-prediction}

Lastly, the metrics of growth as well as the partitioned data will be
used to create a prediction model for the short term growth of research
subfields. In correspondence with the data partitioning, separate models
will be constructed for the statistical research and the computational
research. The objective here is to produce a time series model for short
term projections of growth in research interests and outputs.

\subsection{Limitations}\label{limitations}

Focusing on primarily numerical data as a sign of growth indicates a
relatively simple way of quantifying growth. In reality, growth is a
more complex idea and could benefit from the use of paper content for
text data processing to supplement the numerical figures of growth. This
is a potential avenue of further exploration and work.

\section{Results}\label{sec-results}

\subsection{Exploratory Data Analysis}\label{exploratory-data-analysis}

\subsubsection{Original Data}\label{original-data}

Before examining each partition of the original data for the purpose of
directly answering the research questions, it is important to understand
the context behind and the general appearance of the original dataset
itself. For this reason, many plots were created to visualize parts of
the data during the data pre-processing and data partitioning stages.

\begin{figure}[H]

{\centering \pandocbounded{\includegraphics[keepaspectratio]{images/ai_plot-2.png}}

}

\caption{Artifical Intelligence Paper Output Plot}

\end{figure}%

This initial plot of the research paper outputs for the category of
Artifical Intelligence indicates several patterns we will continue to
see in this data. Though Artifical Intelligence is often seen as a
recent breakthrough that has only gotten larger year by year, we can see
here that the yearly trend is far from consistent.

From here, we then further explore each of the partitioned datasets
rather than focusing on the original alone, as these will be the
foundation of our future analysis and modeling.

\subsubsection{Statistical Data}\label{statistical-data}

The statistical data subset is comprised of the topics Data Analysis,
Statistics and Probability, Machine Learning (Statistics), Methodology
(Statistics), Computation (Statistics), Other Statistics, Applications
(Statistics), and Statistics Theory. In order to explore the relative
frequencies of paper output by these topics, we present an initial plot
of this data.

\begin{figure}[H]

{\centering \pandocbounded{\includegraphics[keepaspectratio]{images/stat_edaplot-2.png}}

}

\caption{Statistical Data Paper Output Plot}

\end{figure}%

From this initial plot it is immediately apparent that Machine Learning
(Statistics) and Methodology appear to be the most popular subdomains,
with other subdomains varying and not having a clear edge over each
other. This provides us with a general impression of the data prior to
delving into the specific yearly figures and relative frequencies.

\section{Verifications}\label{sec-verify}

This section will be just long enough to illustrate what a full page of
text looks like, for margins and spacing.

\addtolength{\textheight}{.5in}%

\citet{gelm:veht:2021} offer some guidance about key ideas about
statistical ideas. On an unrelated note, spreadsheets are important to
use correctly \citep{brom:woo:2018}. Log-linear models are an attractive
way to model categorical data \citep{bish:fien:1975}.

The quick brown fox jumped over the lazy dog. The quick brown fox jumped
over the lazy dog. The quick brown fox jumped over the lazy dog. The
quick brown fox jumped over the lazy dog. \textbf{With this spacing we
have 25 lines per page.} The quick brown fox jumped over the lazy dog.
The quick brown fox jumped over the lazy dog. The quick brown fox jumped
over the lazy dog. The quick brown fox jumped over the lazy dog. The
quick brown fox jumped over the lazy dog.

The quick brown fox jumped over the lazy dog. The quick brown fox jumped
over the lazy dog. The quick brown fox jumped over the lazy dog. The
quick brown fox jumped over the lazy dog. The quick brown fox jumped
over the lazy dog. The quick brown fox jumped over the lazy dog. The
quick brown fox jumped over the lazy dog. The quick brown fox jumped
over the lazy dog. The quick brown fox jumped over the lazy dog. The
quick brown fox jumped over the lazy dog.

The quick brown fox jumped over the lazy dog. The quick brown fox jumped
over the lazy dog. The quick brown fox jumped over the lazy dog. The
quick brown fox jumped over the lazy dog. The quick brown fox jumped
over the lazy dog. The quick brown fox jumped over the lazy dog. The
quick brown fox jumped over the lazy dog. The quick brown fox jumped
over the lazy dog. The quick brown fox jumped over the lazy dog. The
quick brown fox jumped over the lazy dog.

The quick brown fox jumped over the lazy dog. The quick brown fox jumped
over the lazy dog. The quick brown fox jumped over the lazy dog. The
quick brown fox jumped over the lazy dog. The quick brown fox jumped
over the lazy dog. The quick brown fox jumped over the lazy dog. The
quick brown fox jumped over the lazy dog. The quick brown fox jumped
over the lazy dog. The quick brown fox jumped over the lazy dog. The
quick brown fox jumped over the lazy dog.

\addtolength{\textheight}{-.5in}%

\addtolength{\textheight}{.2in}%

The quick brown fox jumped over the lazy dog. The quick brown fox jumped
over the lazy dog. The quick brown fox jumped over the lazy dog. The
quick brown fox jumped over the lazy dog. The quick brown fox jumped
over the lazy dog. The quick brown fox jumped over the lazy dog. The
quick brown fox jumped over the lazy dog. The quick brown fox jumped
over the lazy dog. The quick brown fox jumped over the lazy dog. The
quick brown fox jumped over the lazy dog.

The quick brown fox jumped over the lazy dog. The quick brown fox jumped
over the lazy dog. The quick brown fox jumped over the lazy dog. The
quick brown fox jumped over the lazy dog. The quick brown fox jumped
over the lazy dog. The quick brown fox jumped over the lazy dog. The
quick brown fox jumped over the lazy dog. The quick brown fox jumped
over the lazy dog. The quick brown fox jumped over the lazy dog. The
quick brown fox jumped over the lazy dog.

The quick brown fox jumped over the lazy dog. The quick brown fox jumped
over the lazy dog. The quick brown fox jumped over the lazy dog. The
quick brown fox jumped over the lazy dog. The quick brown fox jumped
over the lazy dog. The quick brown fox jumped over the lazy dog. The
quick brown fox jumped over the lazy dog. The quick brown fox jumped
over the lazy dog. The quick brown fox jumped over the lazy dog. The
quick brown fox jumped over the lazy dog.

The quick brown fox jumped over the lazy dog. The quick brown fox jumped
over the lazy dog. The quick brown fox jumped over the lazy dog. The
quick brown fox jumped over the lazy dog. The quick brown fox jumped
over the lazy dog. The quick brown fox jumped over the lazy dog. The
quick brown fox jumped over the lazy dog. The quick brown fox jumped
over the lazy dog. The quick brown fox jumped over the lazy dog. The
quick brown fox jumped over the lazy dog.

The quick brown fox jumped over the lazy dog. The quick brown fox jumped
over the lazy dog. The quick brown fox jumped over the lazy dog. The
quick brown fox jumped over the lazy dog. The quick brown fox jumped
over the lazy dog. The quick brown fox jumped over the lazy dog. The
quick brown fox jumped over the lazy dog. The quick brown fox jumped
over the lazy dog. The quick brown fox jumped over the lazy dog. The
quick brown fox jumped over the lazy dog.

The quick brown fox jumped over the lazy dog. The quick brown fox jumped
over the lazy dog. The quick brown fox jumped over the lazy dog. The
quick brown fox jumped over the lazy dog. The quick brown fox jumped
over the lazy dog. The quick brown fox jumped over the lazy dog. The
quick brown fox jumped over the lazy dog. The quick brown fox jumped
over the lazy dog. The quick brown fox jumped over the lazy dog. The
quick brown fox jumped over the lazy dog.

The quick brown fox jumped over the lazy dog. The quick brown fox jumped
over the lazy dog. The quick brown fox jumped over the lazy dog. The
quick brown fox jumped over the lazy dog.

\addtolength{\textheight}{-.2in}%

\section{Conclusion}\label{sec-conc}

\section{Disclosure statement}\label{disclosure-statement}

The authors have the following conflicts of interest to declare (or
replace with a statement that no conflicts of interest exist).

\section{Data Availability Statement}\label{data-availability-statement}

Deidentified data have been made available at the following URL: XX.

\phantomsection\label{supplementary-material}
\bigskip

\begin{center}

{\large\bf SUPPLEMENTARY MATERIAL}

\end{center}

\begin{description}
\item[Title:]
Brief description. (file type)
\item[R-package for MYNEW routine:]
R-package MYNEW containing code to perform the diagnostic methods
described in the article. The package also contains all datasets used as
examples in the article. (GNU zipped tar file)
\item[HIV data set:]
Data set used in the illustration of MYNEW method in
Section~\ref{sec-verify} (.txt file).
\end{description}


  \bibliography{bibliography.bib}



\end{document}
