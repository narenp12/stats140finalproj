% Options for packages loaded elsewhere
\PassOptionsToPackage{unicode}{hyperref}
\PassOptionsToPackage{hyphens}{url}
\PassOptionsToPackage{dvipsnames,svgnames,x11names}{xcolor}
%
\documentclass[
  12pt]{article}

\usepackage{amsmath,amssymb}
\usepackage{iftex}
\ifPDFTeX
  \usepackage[T1]{fontenc}
  \usepackage[utf8]{inputenc}
  \usepackage{textcomp} % provide euro and other symbols
\else % if luatex or xetex
  \usepackage{unicode-math}
  \defaultfontfeatures{Scale=MatchLowercase}
  \defaultfontfeatures[\rmfamily]{Ligatures=TeX,Scale=1}
\fi
\usepackage{lmodern}
\ifPDFTeX\else  
    % xetex/luatex font selection
\fi
% Use upquote if available, for straight quotes in verbatim environments
\IfFileExists{upquote.sty}{\usepackage{upquote}}{}
\IfFileExists{microtype.sty}{% use microtype if available
  \usepackage[]{microtype}
  \UseMicrotypeSet[protrusion]{basicmath} % disable protrusion for tt fonts
}{}
\makeatletter
\@ifundefined{KOMAClassName}{% if non-KOMA class
  \IfFileExists{parskip.sty}{%
    \usepackage{parskip}
  }{% else
    \setlength{\parindent}{0pt}
    \setlength{\parskip}{6pt plus 2pt minus 1pt}}
}{% if KOMA class
  \KOMAoptions{parskip=half}}
\makeatother
\usepackage{xcolor}
\setlength{\emergencystretch}{3em} % prevent overfull lines
\setcounter{secnumdepth}{5}
% Make \paragraph and \subparagraph free-standing
\makeatletter
\ifx\paragraph\undefined\else
  \let\oldparagraph\paragraph
  \renewcommand{\paragraph}{
    \@ifstar
      \xxxParagraphStar
      \xxxParagraphNoStar
  }
  \newcommand{\xxxParagraphStar}[1]{\oldparagraph*{#1}\mbox{}}
  \newcommand{\xxxParagraphNoStar}[1]{\oldparagraph{#1}\mbox{}}
\fi
\ifx\subparagraph\undefined\else
  \let\oldsubparagraph\subparagraph
  \renewcommand{\subparagraph}{
    \@ifstar
      \xxxSubParagraphStar
      \xxxSubParagraphNoStar
  }
  \newcommand{\xxxSubParagraphStar}[1]{\oldsubparagraph*{#1}\mbox{}}
  \newcommand{\xxxSubParagraphNoStar}[1]{\oldsubparagraph{#1}\mbox{}}
\fi
\makeatother


\providecommand{\tightlist}{%
  \setlength{\itemsep}{0pt}\setlength{\parskip}{0pt}}\usepackage{longtable,booktabs,array}
\usepackage{calc} % for calculating minipage widths
% Correct order of tables after \paragraph or \subparagraph
\usepackage{etoolbox}
\makeatletter
\patchcmd\longtable{\par}{\if@noskipsec\mbox{}\fi\par}{}{}
\makeatother
% Allow footnotes in longtable head/foot
\IfFileExists{footnotehyper.sty}{\usepackage{footnotehyper}}{\usepackage{footnote}}
\makesavenoteenv{longtable}
\usepackage{graphicx}
\makeatletter
\newsavebox\pandoc@box
\newcommand*\pandocbounded[1]{% scales image to fit in text height/width
  \sbox\pandoc@box{#1}%
  \Gscale@div\@tempa{\textheight}{\dimexpr\ht\pandoc@box+\dp\pandoc@box\relax}%
  \Gscale@div\@tempb{\linewidth}{\wd\pandoc@box}%
  \ifdim\@tempb\p@<\@tempa\p@\let\@tempa\@tempb\fi% select the smaller of both
  \ifdim\@tempa\p@<\p@\scalebox{\@tempa}{\usebox\pandoc@box}%
  \else\usebox{\pandoc@box}%
  \fi%
}
% Set default figure placement to htbp
\def\fps@figure{htbp}
\makeatother

\addtolength{\oddsidemargin}{-.5in}%
\addtolength{\evensidemargin}{-1in}%
\addtolength{\textwidth}{1in}%
\addtolength{\textheight}{1.7in}%
\addtolength{\topmargin}{-1in}%
\makeatletter
\@ifpackageloaded{caption}{}{\usepackage{caption}}
\AtBeginDocument{%
\ifdefined\contentsname
  \renewcommand*\contentsname{Table of contents}
\else
  \newcommand\contentsname{Table of contents}
\fi
\ifdefined\listfigurename
  \renewcommand*\listfigurename{List of Figures}
\else
  \newcommand\listfigurename{List of Figures}
\fi
\ifdefined\listtablename
  \renewcommand*\listtablename{List of Tables}
\else
  \newcommand\listtablename{List of Tables}
\fi
\ifdefined\figurename
  \renewcommand*\figurename{Figure}
\else
  \newcommand\figurename{Figure}
\fi
\ifdefined\tablename
  \renewcommand*\tablename{Table}
\else
  \newcommand\tablename{Table}
\fi
}
\@ifpackageloaded{float}{}{\usepackage{float}}
\floatstyle{ruled}
\@ifundefined{c@chapter}{\newfloat{codelisting}{h}{lop}}{\newfloat{codelisting}{h}{lop}[chapter]}
\floatname{codelisting}{Listing}
\newcommand*\listoflistings{\listof{codelisting}{List of Listings}}
\makeatother
\makeatletter
\makeatother
\makeatletter
\@ifpackageloaded{caption}{}{\usepackage{caption}}
\@ifpackageloaded{subcaption}{}{\usepackage{subcaption}}
\makeatother

\usepackage[]{natbib}
\bibliographystyle{agsm}
\usepackage{bookmark}

\IfFileExists{xurl.sty}{\usepackage{xurl}}{} % add URL line breaks if available
\urlstyle{same} % disable monospaced font for URLs
\hypersetup{
  pdftitle={Research Proposal for ``Research on Research on Research: Analyzing historical trends in statistical and computational research from 1993 to 2024''},
  pdfauthor={Naren Prakash},
  colorlinks=true,
  linkcolor={blue},
  filecolor={Maroon},
  citecolor={Blue},
  urlcolor={Blue},
  pdfcreator={LaTeX via pandoc}}



\begin{document}


\def\spacingset#1{\renewcommand{\baselinestretch}%
{#1}\small\normalsize} \spacingset{1}


%%%%%%%%%%%%%%%%%%%%%%%%%%%%%%%%%%%%%%%%%%%%%%%%%%%%%%%%%%%%%%%%%%%%%%%%%%%%%%

\date{February 25, 2025}
\title{\bf Research Proposal for ``Research on Research on Research:
Analyzing historical trends in statistical and computational research
from 1993 to 2024''}
\author{
Naren Prakash\\
Department of Statistics and Data Science, University of California, Los
Angeles\\
}
\maketitle

\bigskip
\bigskip
\begin{abstract}
This paper aims to analyze the changes in research paper output for
different statistical and computational fields over the time period from
1993 to 2024. The research papers used for this analysis are sourced
from a dataset of papers from the pre-print journal arXiv.
\end{abstract}


\newpage
\spacingset{1.9} % DON'T change the spacing!


\section{Introduction}\label{sec-intro}

In recent years, with the fields of artifical intelligence and machine
learning becoming important parts of the public lexicon and increasingly
becoming involved in our day to day lives, we've seen firsthand large
changes in statistical and computational research. With statistical
methods increasingly becoming intertwined with computational principles,
such as its integration with aspects of computer science, the future of
statistics and computation appear to be one and the same. How does this
current research landscape compare with that of the landscape a mere 30
years ago? This paper aims to analyze historical trends in statistical
and computational research, as tracked by papers submitted to the online
pre-print journal arXiv, in order to visualize the dramatic changes
we've seen over the years and find any subfields growing in the present
that could yet transform the landscape of the future. This analysis of
historical trends will be conducted using a specific dataset available
on Kaggle \citet{arXiv:kaggle:data}.

\section{Research Questions}\label{sec-questions}

\begin{itemize}
\tightlist
\item
  What statistical and computational fields have seen the largest
  increase in publications?
\item
  How have the most published statistical and computational fields
  changed over time?
\item
  What statistical and computational fields are projected to grow the
  most in the coming years?
\end{itemize}

\section{Literature Review}\label{lit-review}

Looking at themes in statistical and computational research is nothing
new. For instance, \citet{gelm:veht:2021} analyzed the dominant
statistical ideas of the past 50 years, suggesting inferential methods,
computational algorithms, and data nalysis have been the most impactful
in the shifting of the research landscape. Smaller subsets of time have
also been analyzed, with \citet{jun:yoo:choi:2018} using Google Trends
to track the growth of different subfields of research (with an emphasis
on big data and application). This paper aims to look at a similar
problem with a different lens, using publication outputs themselves as a
way of analyzing changes in research focus and interest. In doing this,
the paper aims to also obtain an indication of the subfields with
increasing research interest in the short term that may lend itself to
future publications. Evaluating the research trends of the future has
often involved modeling itself, such as the hype cycle model
\citet{dedehayir:2016}. This model aims to track the life cycle of
technological innovations. In a similar vein, this paper aims to use
current and recent paper production output to indicate trends of the
near future. As the methodology involves using the research paper
pre-prints themselves, it may provide a clearer picture of specific
publication interests and trends rather than topics and concepts in
general.

\section{Methods}\label{sec-meth}

\subsection{Data Parti}\label{data-parti}


  \bibliography{bibliography.bib}



\end{document}
